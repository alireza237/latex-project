\documentclass{beamer}
\usepackage[utf8]{inputenc}
\usetheme{Warsaw}
\usecolortheme{beaver}
%Information to be included in the title page:


\title{Linear algebra}
\author{Alireza Qanbari}
%\institute{Overleaf}
\date{\today}






\begin{document}
    \frame{\titlepage}




    \begin{frame}
        \frametitle{Contents}
        \tableofcontents[pausesections]
    \end{frame}


    \section{Vector spaces}

    \begin{frame}
        \frametitle{Vector spaces}
        A vector space over a field F (often the field of the real numbers) is a set $V$ equipped with two binary operations
        satisfying the following axioms. Elements of $V$ are called vectors, and elements of $F$ are called scalars.\cite{2}
        \begin{itemize}


            \item<1->The first operation, vector addition, takes any two vectors v and w and outputs a third vector v + w.
            \item <2->The second operation, scalar multiplication, takes any scalar a and any vector v and outputs a new vector av.

        \end{itemize}
    \end{frame}

    \begin{frame}
        \frametitle{Vector spaces}
        \begin{center}

            \begin{tabular}{||l | l||}
                \hline
                Axiom & Signification \\
                \hline
                Associativity of addition &  $u+(v+w)=(u+v)+w$  \\
                \hline
                Commutativity of addition & $u+v=v+u$ \\
                \hline
                Identity element of addition & There exists an element 0 in V,\\
                &called the zero vector (or simply zero),\\
                & such that v + 0 = v for all v in V. \\
                \hline


            \end{tabular}
        \end{center}


    \end{frame}

    \subsection{Linear maps}

    \begin{frame}
        \frametitle{Linear maps}



        \begin{block}{Linear maps}
            Linear maps are mappings between vector spaces that preserve the vector-space structure. Given two vector
            spaces V and W over a field F, a linear map (also called, in some contexts, linear transformation or linear mapping)
            is a map

        \end{block}
        \pause

        \begin{alertblock}{}
            $$T:V\to W$$
        \end{alertblock}
        \pause

        that is compatible with addition and scalar multiplication, that is
        for any vectors u,v in V and scalar a in F.
        \pause

        \begin{alertblock}{}


            $$T(u+v)=T(u)+T(v),\quad T(av)=aT(v)$$

        \end{alertblock}


    \end{frame}


    \begin{frame}
        \frametitle{Linear maps}
        This implies that for any vectors u, v in V and scalars a, b in F, one has
        When $V = W$ are the same vector space, a linear map $T:V\to V$ is also known as a linear operator on $V$
        \pause

        \begin{alertblock}{}
            $$T(au+bv)=T(au)+T(bv)=aT(u)+bT(v)$$

        \end{alertblock}

    \end{frame}

    \subsection{Subspaces, span, and basis}

    \begin{frame}
        \frametitle{Subspaces, span, and basis}
        The study of those subsets of vector spaces that are in themselves vector spaces under the induced operations is
        fundamental, similarly as for many mathematical structures. These subsets are called linear subspaces. More
        precisely, a linear subspace of a vector space $V$ over a field $F$ is a subset $W$ of $V$ such that $u+v$ and au are
        in $W$, for every $u, v$ in W, and every $a$ in $F$.
        \pause

        \begin{examples}
            given a linear map $T:V\to W$ , the image $T(V)$ of $V$, and the inverse image $ T^{-1}(0)$
            of 0 (called kernel or null space), are linear subspaces of $W$ and $V$, respectively.

        \end{examples}


    \end{frame}

    \begin{frame}
        \frametitle{Subspaces, span, and basis}
        Another important way of forming a subspace is to consider linear combinations of a set S of vectors: the set of
        all sums
        \pause
        \begin{alertblock}{}
            $$ a_{1}v_{1}+a_{2}v_{2}+\cdots +a_{k}v_{k}, $$

        \end{alertblock}
        \pause

        \begin{block}{}
            where $v_{1}+v_{2}+\cdots +v_{k}$ are in $S$, and $a_{1}+a_{2}+\cdots +a_{k}$, ak are in $F$ form a linear subspace
            called the span of $S$. The span of S is also the
            intersection of all linear subspaces containing $S$. In other words, it is the (smallest for the inclusion relation)
            linear subspace containing $S$.

        \end{block}


    \end{frame}


    \begin{frame}
        \frametitle{Subspaces, span, and basis}
        \begin{block}{spanning set}
            A set of vectors that spans a vector space is called a spanning set or generating set.\cite{3}

        \end{block}
        \pause

        \begin{block}{basis of}
            Such a linearly independent set that spans a vector space $V$ is called a basis of $V$

        \end{block}
        \pause

        if $S$ is a linearly independent set,
        and $T$ is a spanning set such that $S\subseteq T,$ then there is a basis B such that

        \pause

        \begin{alertblock}{}
            $S\subseteq B\subseteq T.$

        \end{alertblock}

    \end{frame}

    \begin{frame}
        \frametitle{Subspaces, span, and basis}



        \begin{block}{}

            If any basis of $V$ (and therefore every basis) has a finite number of elements, $V$ is a finite-dimensional vector
            space. If $U$ is a subspace of $V$, then $dim U \leq dim V$. In the case where $V$ is finite-dimensional, the equality of
            the dimensions implies $U=V$.

        \end{block}
        \pause


        \begin{alertblock}{}
            If $U_1$ and $U_2$ are subspaces of $V$, then

            $$\dim(U_{1}+U_{2})=\dim U_{1}+\dim U_{2}-\dim(U_{1}\cap U_{2}),$$

            where $U_{1}+U_{2}$ denotes the span of $ U_{1}\cup U_{2}.$\cite{4}

        \end{alertblock}

    \end{frame}


    \section{References}
    \begin{frame}
        \frametitle{References}
        \begin{thebibliography}{1}

            \bibitem{1} {
                Benjamin Peirce (1872) Linear Associative Algebra, lithograph, new edition with corrections, notes,
                and an added 1875 paper by Peirce, plus notes by his son Charles Sanders Peirce, published in
                the American Journal of Mathematics v. 4, 1881, Johns Hopkins University, pp. 221 226, Google Eprint}
            \bibitem{2} {
                and as an extract, D. Van Nostrand, 1882, Google Eprint.}
            \bibitem{3} {
                Roman (2005, ch. 1, p. 27)}
            \bibitem{4} {
                Axler (2004, p. 55)}
        \end{thebibliography}

    \end{frame}

\end{document}